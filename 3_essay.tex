\documentclass[main.tex]{subfiles}

\begin{document}
\section{Реферат}

Курсовий проект присвячено розробці та розрахунку підсилювача низької частоти з заданими технічними характеристиками. Метою роботи є набуття практичних навичок проектування електронних пристроїв, розрахунку їх компонентів та аналізу роботи схем підсилення.

\textbf{Об'єкт дослідження:} підсилювач низької частоти на основі дискретних компонентів з інтегральним стабілізатором напруги живлення.

\textbf{Предмет дослідження:} методи розрахунку каскадів підсилення, вибір елементної бази, аналіз частотних характеристик та забезпечення стабільного живлення схеми.

У роботі виконано:
\begin{itemize}
    \item Попередній розрахунок структурної схеми трикаскадного підсилювача низької частоти;
    \item Вибір типів транзисторів для каскадів попереднього підсилення (КТ315Г) та кінцевого каскаду (КТ815Г, КТ814Г);
    \item Детальний розрахунок каскаду попереднього підсилення зі спільним емітером;
    \item Розрахунок номіналів пасивних компонентів схеми;
    \item Аналіз амплітудно-частотної характеристики в діапазоні частот 100 Гц -- 16 кГц;
    \item Вибір та розрахунок інтегрального стабілізатора напруги КР142ЕН8В для живлення підсилювача.
\end{itemize}

\textbf{Основні результати:}
\begin{enumerate}
    \item Розроблено схему електричну принципову підсилювача низької частоти з вихідною потужністю 1.4 Вт на навантаженні 6 Ом;
    \item Забезпечено коефіцієнт підсилення за потужністю 27.1 дБ для каскаду попереднього підсилення;
    \item Досягнуто нерівномірність АЧХ в робочому діапазоні частот не більше 0.8 дБ;
    \item Розраховано параметри всіх компонентів схеми з урахуванням стандартних номіналів;
    \item Забезпечено стабільне живлення схеми напругою 14 В при зміні вхідної напруги в межах ±20\%.
\end{enumerate}

Розроблений підсилювач може бути використаний у системах звуковідтворення, аудіоапаратурі та інших пристроях, де потрібне підсилення сигналів звукової частоти з високою якістю.

\end{document}