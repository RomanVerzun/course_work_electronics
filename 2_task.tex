\documentclass[main.tex]{subfiles}

\begin{comment}
    Завдання на курсовий проект, зміст розрахунково-пояснювальної записки,
перелік графічного матеріалу видається студенту керівником.
\end{comment}

\begin{document}
\newpage
\section{Завдання}

\subsection{Тема курсового проекту}
Розрахунок та проектування транзисторного підсилювача низької частоти.

\subsection{Вихідні дані для проектування}
\begin{enumerate}
    \item Вихідна потужність підсилювача: $P_\text{вих} = 2$ Вт
    \item Опір навантаження: $R_\text{н} = 4$ Ом
    \item Напруга джерела вхідного сигналу: $U_\text{вх} = 15$ мВ
    \item Внутрішній опір джерела вхідного сигналу: $R_{\text{дж}} = 330$ Ом
    \item Вид схеми: T (з трансформаторним зв'язком)
    \item Діапазон робочих частот: $f_{\text{н}} - f_{\text{в}} = 50 - 20000$ Гц
    \item Температура оточуючого середовища: $T_{min} = 15^\circ C$; $T_{max} = 25^\circ C$
\end{enumerate}

\subsection{Зміст розрахунково-пояснювальної записки}
\begin{enumerate}
    \item \textbf{Попередній (ескізний) розрахунок ПНЧ:}
    \begin{itemize}
        \item Розрахунок потужності вхідного сигналу
        \item Визначення необхідного коефіцієнта підсилення за потужністю
        \item Вибір схеми та типу транзисторів вихідного каскаду
        \item Визначення орієнтовної кількості каскадів підсилення
        \item Розробка структурної схеми ПНЧ
        \item Складання орієнтовної принципової схеми ПНЧ
    \end{itemize}
    
    \item \textbf{Остаточний розрахунок каскаду попереднього підсилення:}
    \begin{itemize}
        \item Вибір та обґрунтування типу транзистора
        \item Розрахунок режимів роботи транзистора по постійному струму
        \item Розрахунок елементів схеми (резисторів базового дільника, колекторного та емітерного навантаження)
        \item Розрахунок розділових та блокувальних конденсаторів
        \item Визначення коефіцієнтів підсилення каскаду
        \item Побудова амплітудно-частотної характеристики
    \end{itemize}
    
    \item \textbf{Вибір та розрахунок інтегрального стабілізатора напруги:}
    \begin{itemize}
        \item Визначення параметрів джерела живлення
        \item Вибір типу інтегрального стабілізатора
        \item Розрахунок елементів схеми стабілізатора
        \item Визначення ККД та теплового режиму
    \end{itemize}
\end{enumerate}

\subsection{Перелік графічного матеріалу}
\begin{enumerate}
    \item Структурна схема підсилювача низької частоти
    \item Принципова електрична схема ПНЧ
    \item Принципова електрична схема каскаду попереднього підсилення
    \item Амплітудно-частотні характеристики каскаду
    \item Схема джерела живлення з інтегральним стабілізатором
\end{enumerate}
\end{document}