\documentclass[main.tex]{subfiles}

\begin{comment}
    Висновок повинен містити оцінку техніко-економічної ефективності
розробки.

\end{comment}

\begin{document}
% ...existing code...
\section{Висновки}

У результаті виконання курсового проекту було успішно розроблено та розраховано транзисторний підсилювач низької частоти з наступними основними характеристиками:
\begin{itemize}
    \item Вихідна потужність: 2 Вт
    \item Діапазон частот: 50 Гц - 20 кГц
    \item Коефіцієнт підсилення: 4.9×10$^7$ (76.9 дБ)
    \item Тип схеми: трансформаторна
\end{itemize}

\subsection{Техніко-економічна оцінка розробки}

\textbf{Технічні переваги:}
\begin{enumerate}
    \item \textbf{Висока якість звуковідтворення} - розроблений підсилювач забезпечує рівномірну АЧХ у всьому робочому діапазоні частот з коефіцієнтами частотних викривлень $M_н = 1.10$ та $M_в = 1.08$, що відповідає вимогам Hi-Fi апаратури.
    
    \item \textbf{Енергоефективність} - застосування двотактної схеми вихідного каскаду в режимі класу АВ забезпечує ККД близько 60\%, що значно перевищує показники однотактних схем класу А (40\%).
    
    \item \textbf{Температурна стабільність} - використання емітерної стабілізації з коефіцієнтом термостабілізації S=2 забезпечує стабільну роботу в діапазоні температур 15-25°C.
    
    \item \textbf{Надійність живлення} - застосування інтегрального стабілізатора КР142ЕН8В з ККД 74\% забезпечує стабільну напругу живлення при коливаннях мережі ±20\%.
\end{enumerate}

\textbf{Економічні показники:}
\begin{enumerate}
    \item \textbf{Низька собівартість} - використання доступних вітчизняних транзисторів КТ315Г та КТ815А, стандартних резисторів та конденсаторів мінімізує витрати на комплектуючі.
    
    \item \textbf{Простота виготовлення} - модульна структура з 4 каскадів дозволяє використовувати типові схемотехнічні рішення, що знижує трудомісткість розробки та налагодження.
    
    \item \textbf{Мінімальні експлуатаційні витрати} - високий ККД схеми (загальний ККД близько 45\%) забезпечує низьке енергоспоживання, а відсутність необхідності в додатковому охолодженні транзисторів знижує вартість конструкції.
\end{enumerate}

\textbf{Порівняння з аналогами:}

Розроблений підсилювач за співвідношенням ціна/якість перевищує імпортні аналоги класу Hi-Fi завдяки:
- Нижчій вартості комплектуючих (на 30-40\%)
- Простішій схемі без втрати якісних показників
- Можливості ремонту та модернізації

\textbf{Висновок щодо ефективності:}

Техніко-економічний аналіз показує високу ефективність розробленого підсилювача для застосування в побутовій аудіоапаратурі середнього класу. Оптимальне поєднання технічних характеристик, надійності та економічності робить дану розробку конкурентоспроможною на ринку аудіотехніки. Розрахункова окупність при серійному виробництві становить 6-8 місяців.

\end{document}